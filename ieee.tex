\documentclass[conference]{IEEEtran}
\IEEEoverridecommandlockouts

% Package
\usepackage{cite}
\usepackage{amsmath,amssymb,amsfonts}
\usepackage{algorithm}
\usepackage[noend]{algpseudocode}
\usepackage{xcolor}
\usepackage{graphicx}
\usepackage{textcomp}
\usepackage{xcolor}

% Settings
%   1
\def\BibTeX{{\rm B\kern-.05em{\sc i\kern-.025em b}\kern-.08em
    T\kern-.1667em\lower.7ex\hbox{E}\kern-.125emX}}
%   2
\makeatletter
%	3
\newcommand{\linebreakand}{
  \end{@IEEEauthorhalign}
  \hfill\mbox{}\par
  \mbox{}\hfill\begin{@IEEEauthorhalign}
}
%   4
\makeatother
%   5
\pagecolor{white}

% ----- DOCUMENT STARTS HERE -----
\begin{document}

\title{Petification: Node-RED based Pet Care IoT Solution \\ using MQTT Broker}

\author{
    \IEEEauthorblockN{Haeram Kim}
    \IEEEauthorblockA{
        \textit{Computer Science and Engineering} \\
        \textit{Chungnam National University}\\
        Daejeon, Korea \\
        haeram.kim1@gmail.com
    }
    \and
    \IEEEauthorblockN{Hyejong Kang}
    \IEEEauthorblockA{
        \textit{Computer Science and Engineering} \\
        \textit{Chungnam National University}\\
        Daejeon, Korea \\
        kanghyejong1001@gmail.com
    }
    \and
    \IEEEauthorblockN{Sunghan Kim}
        \IEEEauthorblockA{
        \textit{Computer Science and Engineering} \\
        \textit{Chungnam National University}\\
        Daejeon, Korea \\
        seonghan.kim.cnu@gmail.com
    }
    \linebreakand
    \IEEEauthorblockN{Dukho Choi}
    \IEEEauthorblockA{
        \textit{International trade / software convergence} \\
        \textit{Chungnam National University}\\
        Daejeon, Korea \\
        dukho.fin@gmail.com
    }
    \and
    \IEEEauthorblockN{Jihyun You}
    \IEEEauthorblockA{
        \textit{Cybersecurity} \\
        \textit{Purdue University}\\
        West Lafayette, IN, USA \\
        you62@purdue.edu
    }
}

\maketitle 

% ----- ABSTRACT -----
\begin{abstract}
While there are an increasing number of households owning pets, it is challenging for owners who leave home often to take good care of their pets. However, most of the previous studies which are conducted to serve food and water remotely use the limited featured IoT platforms and do not provide remaining amount of the food and water and empty notification service to the user. The proposed IoT solution named "Petification" uses open-source project Node-RED with MQTT messaging protocol to provide information about device connectivity and remaining amount of food and water as well as consumption via web-based dashboard. Additionally, Petification provides an empty notification service among with remote feeding service. The water supplier and feed machine are attached to the platform to provide water and food to the pet and scale the weight of the water and food. The load cell, HX711 amplifier, and Raspberry Pi Zero W are mounted to the water supplier and feed machine and MG90S servo motor is mounted to Raspberry Pi of the feed machine. However, served amount of the food sometimes mismatches with the desired amount whlie testing the implementation. Thus the future plan can be enhancing the food gate to serve exact amount of food. \\
\end{abstract}

% ----- KEYWORD -----
\begin{IEEEkeywords}
IoT platform, Node-RED, MQTT, Smart pet care service 
\end{IEEEkeywords}

% —— INTRODUCTION ——
\section{Introduction}
% `Project background`
As the number of households living alone increases and the culture of raising pets spreads compared to the past, the number of households raising pets is increasing. This trend is shown in the growth of the pet industry; The profit of the pet industry is more than doubled every year over 10 years, from \$48.4 billion in 2010 to \$109.6 billion in 2020 \cite{b1}. Along with this trend, demand for tracking pet wellness is increased accordingly. One of the demanded services is tracking a pet’s status when the pet is left alone. Because for those who left home often, it is challenging to take good care of their pets.

% `Problem statement`
As a way to solve this problem, Internet of Things (IoT) technology has emerged. A lot of pet care IoT solutions are introduced in the market, and many studies and implementations are suggested.
However, the IoT platforms which are used to implement pet care IoT solution has limitations. For example, a most commonly used IoT platform is "Blynk" \cite{b2, b3, b4, b5}, but the latest version of it is not an open-source and open-source version is no longer maintained. Also, "Adafruit IO" \cite{b6} and "Freeboard IO" \cite{b7} are another commonly used IoT platform, but they are not open-source.
Moreover, while tracking water and food consumption and automatic feeding are commonly supported by previous implementations, there are few solutions that support device status information and error notification.

% `Project novelty`
The proposed IoT solution named "Petification" uses open-source projects for IoT platform. Node-RED which is a flow-based open-source visual programming tool \cite{b8} is used to implement the APIs of the IoT platform. By using Node-RED, Petification can get the benefit of providing user-friendly UI, fast development, and a lot of resources shared by users. Also, Eclipse Mosquitto which is one of the open-source implementation of the MQTT messaging protocol is used to manage overall message flow. Using MQTT protocol as a message broker allows the IoT platform to connect user with devices in lightweight way \cite{b9}.

% `Functionality`
Petification provides information about device connectivity and amount of remaining  food/water for each device as well as food/water consumption. Additionally, Petification provides an error notification service when the water or food is running out of empty among with automatic feeding service.
The water supplier and feed machine are attached to the platform to feed and water to the pet and to scale the weight of the food/water. While both water supplier and feed machine uses the load cell to scale the weight, a servo motor is attached only to the feed machine so that the user can control the served amount of food. All the load cells and a servo motor are mounted to Raspberry Pi for each device.
The web-based dashboard is supported to provide these information and services in the visual way. The food/water empty notification is also provided to the user via email and WhatsApp mobile application.

% —— Literature Review ——
\section{Related Literature}
% Reference 1
P. N. Vrishanka \textit{et al.} \cite{b10} proposed an automated pet feeder which serves food to the pet according to the remaining amount of food in the food bowl. An ultrasonic distance sensor and SG90 servo motor were mounted to Arduino Uno R3 in this research. An ultrasonic distance sensor is used to determine the remaining food amount by measuring the distance from the entrance of the feed container to the inside of the bowl.

% Reference 2
Rogerio Nogueira \textit{et al.} \cite{b11} proposed a system that can serve food and water according to pre-scheduled times and amount. Additionally, the system provides notification service for the abnormal situations with the photos and alerts and "chat-based" intelligent interface to the user. This research is conducted with Raspberry Pi, Python, Telegram cloud service and IBM Watson Assistent.

% Reference 3
Vania \textit{et al.} \cite{b12} proposed an IoT solution that identifies each pet via RFID and serves food by schedule. The dog feeder in the research was developed using Arduino Uno and ESP8266 Wi-Fi Module with load cell and DC motor. An Android mobile application is provided to the user and the IoT platform which is developed with Node JS and MySQL database controls device-user communication using MQTT protocol.

% Reference 4
Y. Chen \textit{et al.} \cite{b5} proposed a pet care IoT solution that provides information such as food and water consumption, the number of defecation, and defecation duration. It was implemented using Arduino Uno and ESP8266 Wi-Fi Module with load cell, servo motor and motion sensor. A mobile application is provided as an UI with the help of “Blynk” IoT platform.

% Reference 5
T. Sangvanloy \textit{et al.} \cite{b4} proposed a pet care IoT solution that visualizes the daily food consumption in real-time and automatically feeds the pet according to the scheduled time. The conducted research also provides automatic serving amount adjustment feature based on the pet's species and weight. The IoT solution was implemented with ESP32 Wi-Fi micro-controller, servo motor, load cell, and “Blynk” IoT platform.

% —— METHODOLOGY ——
\section{Methodology}
The system architecture for Petification is designed to connect the feed machine and water supplier with the users. Each users and devices can communicate bi-directionally through the platform. The feed machine and water supplier are connected to the Access Point (AP) through Wi-Fi. Thus, all the messages published from devices are sent to the AP with Wi-Fi, and then they are sent to the Petification platform through the internet. Petification platform processes the messages and sends the processed data to the users in form of a web dashboard page, E-Mail, and WhatsApp messages. On the contrary, users can send requests with the web dashboard to the platform. While some requests require only the platform, the requests to activate the actuator are forwarded to the device with the help of the platform. The overview for Petification is shown in figure 2.

% `Overview.png`
\begin{figure}[htbp]
\centerline{\includegraphics[width=0.5\textwidth]{./images/Overview.png}}
\caption{Petification overview}
\label{fig}
\end{figure}

\subsection{Water Supplier}
Measuring the weight of the water is done with one load cell which scales up to maximum 5kg with 1g accuracy. A load cell is installed between the two plates to measure the pressure applied to an upper plate as shown in figure 2 and is attached to the Raspberry Pi through the HX711 amplifier which converts the analog signal to the digital signal as shown in figure 3. To convert the measurement of the load cell to the weight data, the calibration process has proceeded to get the reference unit. The process of calibration is shown below:
\[
	referenceUnit = \frac{average(loadCellMeasurements)}{actualWeight}
\]

A 500mL water bottle which is used as actual weight of 500g is measured 500 times to get a reference unit.

\begin{figure}[htbp]
\centerline{\includegraphics[width=0.3\textwidth]{./images/load-cell.png}}
\caption{Installaion of a load cell}
\label{fig}
\end{figure}

\begin{figure}[htbp]
\centerline{\includegraphics[width=0.5\textwidth]{./images/water supplier circuit.jpg}}
\caption{Circuit diagram of the water supplier}
\label{fig}
\end{figure}

\subsection{Feed Machine}
To serve food to the pet, a MG90S servo motor is mounted to open and close the food gate. As in figure 4, the servo motor gives food in an open/close type. Food serving is done with the help of gravity; After the food gate is opened, food rolls down the slope, and thus food is served. By closing the food gate Feed Machine stops the serving of the food. Two load cells and two HX711 amplifiers are installed in the same way as water supplier under the food bowl and container to measure the weight of the food. All the load cells, HX711 amplifiers and a MG90S servo motor are attached to the Raspberry Pi Zero W as shown in figure 5.

\begin{figure}[htbp]
\centerline{\includegraphics[width=0.5\textwidth]{./images/servo_gate.png}}
\caption{Food gate of the feed machine}
\label{fig}
\end{figure}

\begin{figure}[htbp]
\centerline{\includegraphics[width=0.5\textwidth]{./images/feed machine circuit.jpg}}
\caption{Circuit diagram of the feed machine}
\label{fig}
\end{figure}

\subsection{Platform}
The user-to-device communication is occured when the user wants to serve the food to the pet, and device-to-user communication is occured to provide information and notification to the user. To serve the food, user can either set the serving schedule or press the button in the dashboard. The serving schedules are stored to the database and checked every minute to send the message to the feed machine when the scheduled time arrives. The food serving message is also sent when the user presses the button. After the feed machine receives the message, it opens the food gate and scales the food bowl repeatedly to serve desired amount of food to the pet. The detailed flow for the food serving is shown in figure 6.

\begin{figure}[htbp]
\centerline{\includegraphics[width=0.5\textwidth]{./images/user2device.png}}
\caption{Activity diagram for user-to-device communication}
\label{fig}
\end{figure}

To provide information and notification to the user, both water supplier and feed machine scales what they stores. When the measured weight is above certain threshold, they sends message containing the measured weight to the platform. This data message is used to update and show the remaining amount, consumption and device status to the user. However, when the measured weight is below the threshold, then error message is sent to the platform to update the device status and to notify the user that the food/water is empty. The detailed device-to-user flow is shown in figure 7 and the pseudo-code for calculating the consumption is shown in algorithm 1.

\begin{figure}[htbp]
\centerline{\includegraphics[width=0.5\textwidth]{./images/device2user.png}}
\caption{Activity diagram for device-to-user communication}
\label{fig}
\end{figure}

\begin{algorithm}
\caption{Calculate consumption}\label{algo}
\begin{algorithmic}[1]
    \Procedure{CalcConsumption}{}
        \State $prevConsumption \gets \text{previous } \textit{consumption}$
        \State $currScale \gets \textit{scale} \text{ of received message}$
        \State $prevScale \gets \textit{scale} \text{ of  previous record}$
        \State $decrease \gets 0$
        \If{$precScale \neq \textit{null} \And prevScale > currScale$}
            \State $decrease \gets prevScale - currScale$
        \EndIf
        \Return $prevConsumption + decrease$
    \EndProcedure
\end{algorithmic}
\end{algorithm}

Three open-source projects are used in the platform: Node-RED, Eclipse Mosquito, and MySQL.

\subsubsection{Node-RED}
Node-RED is a flow-based oen-source visual development tool that is easy for developers and non-developers to develop programs \cite{b17}.
One of the strength of the Node-RED is powerful community. Not only Node-RED itself but also various nodes such as dashboard widgets and database drivers are being distributed through Node Package Manager (NPM).
Another Node-RED’s strength is that it can run on various environments such as local, Raspberry Pi, Docker, Cloud Instance, \textit{et al.} Thus, Node-RED v2.2.1 is installed in not only the platform server but also Raspberry Pi of water supplier and feed machine.

\subsubsection{Eclipse Mosquitto MQTT message broker}
MQTT (Message Queuing Telemetry Transport) is an OASIS standard protocol and provides light-weight, publish/subscribe messaging transport for IoT \cite{b9}.
The publish/subscribe model of the MQTT protocol allows the IoT platform to support bi-directional communication in a way that one can subscribe the "topic" of the message to receive a message published to that "topic" by another.
In this research, Eclipse Mosquitto v1.4.15 is used as an MQTT Message Broker.
Eclipse Mosquitto is an open-source message broker implementing MQTT protocol versions 5.0, 3.1.1, and 3.1\cite{b20}. 

\subsubsection{MySQL}
MySQL is a fast, flexible, and easy-to-use open-source database with RDBMS (relational database management system).
Among with the performance and security perspective, MySQL is considered a suitable database to manage effective data flows because it has good compatibility with Node-RED \cite{b21}.
In this reason, MySQL v5.7.37 is integrated as a component of the platform.

% —— IMPLEMENTATION ——
\section{Implementation}

% ::Calibration and Weight measurement::
\subsection{Water supplier}
Calibrating and measuring the weight of the water is done with python code. By using the "exec" node of the Node-RED in Raspberry Pi to execute python code, the result of a measurement can be used in Node-RED. However, even if the calibration has proceeded, there is always the possibility of measurement error. Thus the "smooth" node of the "node-red-node-smooth" library has been used to get a maximum value of two measurement results.
After the weight is measured, the "switch" node determines what type of message (weight data or error) has to be sent by checking if the weight is below the threshold value which is 10g for this implementation. The messages are sent after the "function" node prepares the message and the "mqtt out" node publishes the message. The Node-RED flow for publishing weight or error messages is shown in figure 11 and the implemented water supplier is shown in figure 12.

\begin{figure}[htbp]
\centerline{\includegraphics[width=0.5\textwidth]{./images/Water Supplier Error Detection.png}}
\caption{Node-RED screenshot for publishing weight or error message}
\label{fig}
\end{figure}

\begin{figure}[htbp]
\centerline{\includegraphics[width=0.5\textwidth]{./images/water-supplier.jpg}}
\caption{Result of implementation for water supplier}
\label{fig}
\end{figure}

% ::Automatic feeding::
\subsection{Feed machine}
Among with the weight measuring flow which is same as that of the water supplier, flow for serving certain amount of the food to the pet is implemented in the feed machine. Food serving flow does not begin before a message which contains desired serving amount at the message payoad arrives. The "join-wait" node of the "node-red-contrib-join-wait" node module waits a food seving message and returns a object which contains a current weight for that time and desired weight after the serving message is arrived. Then, the current weight is measured and compared repeatedly by a "loop" node of the "node-red-contrib-loop-processing" and a "change" node. The food gate is opened for the first loop and is closed when the current weight is above the desired weight or the loop is repeated 500 times by the "pi-gpiod out" node of the "node-red-node-pi-gpiod" node module. The Node-RED flow for serving food is shown in figure 13 and the implemented feed machine is shown in figure 14.

\begin{figure}[htbp]
\centerline{\includegraphics[width=0.5\textwidth]{./images/automaticFeeding.png}}
\caption{Node-RED screenshot for automatic feeding}
\label{fig}
\end{figure}

\begin{figure}[htbp]
\centerline{\includegraphics[width=0.5\textwidth]{./images/feed-machine.jpg}}
\caption{Result of implementation for feed machine}
\label{fig}
\end{figure}

% ::—— Software Implementation ——::
\subsection{Platform}
The platform for Petification uses cloud instances for deployment. On the cloud instance, Mosquitto, MySQL, and Node-RED are installed and firewall, DNS, and certificate for TLS/SSL communication are configured for networking. In the Node-RED, 7 blocks are implemented as 7 flows. The screenshot for implemented Node-RED flow is shown in Figure 13.

\begin{figure}[htbp]
\centerline{\includegraphics[width=0.5\textwidth]{./images/platformImpl.png}}
\caption{Node-RED screenshot for flow of platform}
\label{fig}
\end{figure}

\subsubsection{MQTT Controller}
MQTT Manager in the Node-RED is the gateway for MQTT messages to enter Node-RED. The main purpose of this block is to convert incoming MQTT messages to give convenience to other Node-RED-based blocks. To achieve it, this block receives all the messages by subscribing to all the MQTT message topics. And then it parses message topic and payload to provide useful information such as the MQTT username and client id. These pieces of information are used in other Node-RED-based nodes in further progress.

\subsubsection{Rule Engine}
The purpose of Rule Engine is to activate actions according to MQTT messages. It cooperates with the Rule Table of Database to activate the action. Designing logic of the Rule Engine is inspired by the book, “Build Your Own IoT Platform” \cite{b22}. For every published message, Rule Engine searches for all rules where the rule’s message pattern satisfies the message content. Actions that corresponded to the rules are defined as ReST API form, thus activating action will be progressed as sending an HTTP request. It also provides ReST APIs for adding, modifying, and deleting rules. Some ReST APIs for actions are defined in another block, whereas some APIs are defined in Rule Engine, such as sending notifications.

\subsubsection{Device Manager}
Handling devices that are attached to the platform by users is the main purpose of the Device Manager block. It provides ReST APIs that can add a new device to the platform (with HTTP POST method), modify the status of the device (with HTTP PATCH method), delete a device (with HTTP DELETE method). And it also provides functionality for publishing an action-type message to a specific device.

\subsubsection{Dashboard}
Dashboard Manager is to provide Graphical User Interface (GUI) to the user. By using this block, users can be provided the status of water and food remaining and consumption visually. It also provides buttons to serve food and input areas to set user settings. The screenshots for implented dashboard are shown in figure 13 and 14.

\begin{figure}[htbp]
\centerline{\includegraphics[width=0.5\textwidth]{./images/feed_machine_ui.png}}
\caption{Screenshot for Feed Machine tab}
\label{fig}
\end{figure}

\begin{figure}[htbp]
\centerline{\includegraphics[width=0.5\textwidth]{./images/water_supplier_ui.png}}
\caption{Screenshot for Water Supplier tab}
\label{fig}
\end{figure}

\subsubsection{User Management}
The purpose of the User Manager block is to provide an interface for modifying user settings. Users can manage these 4 settings: Notification, Email address, WhatsApp information, and timezone where the user lives.  To support global users, Timezone settings are included. As all the times that platform uses and database stores use Coordinated Universal Time (UTC) timezone, converting the time of the user to the UTC is necessary. This setting is especially important when a user adds a new schedule by using Schedule Engine, and when providing local time to the user in the Dashboard.

\subsubsection{Schedule Engine}
Handling and executing schedules is the main functionality for this block. For schedule execution, it cooperates with the Schedule Table in the Database. Every minute, the Schedule Engine checks the Schedule Table and executes the actions that are scheduled to be activated at that time. Also, this block provides ReST APIs that can create, read, update and delete the schedule.

\subsubsection{Timeseries Manager}
Managing Time-series Data Table of Database is the main purpose for Time-series Manager block. This block provides 3 types of ReST APIs for managing time-series data: Inquiring record feature (with HTTP GET method), Changing record validity (with HTTP PATCH method), and Deleting record (with HTTP DELETE method).

% ::—— Dashboard Implementation ——::
%\subsection{Dashboard}
%Petification’s dashboard was implemented using ‘node-red-dashboard’, which provides a set of nodes to make a live data dashboard \cite{b23}.
%The dashboard consists of three tabs: Feed Machine tab, Water Supplier tab, and User Settings tab.

% ::Feed Machine tab::
%\subsubsection{Feed Machine tab}
%In the Feed Machine tab, Device Information widget, Feed Machine Statistics widget, and Serve Food widget are provided. As all the functionalities this tab provides are device-specific, selecting a device is included in the Device Information widget and is displayed as a dropdown.
% ::Device connectivity::
%Also, the status for the selected device is provided in the text and LED-shaped icon. The color of the LED-shaped icon changes to green when the device is connected, red for disconnected status, and yellow for error status.

% ::Remaining of the food::
%In the Feed Machine Statistics widget, the Remaining amount of food and container and food consumption is provided. The remaining amount of food bowl and container is displayed as gauge and food consumption is displayed as a line graph.

% ::Serve food by button or schedule::
%In the Serve Food widget, the user input interface is provided to publish feed serving action by button. The serve Food widget also provides the interface for automatic food feeding (scheduled feeding). Users can see all the feeding schedules in the table format, add a new schedule by user input interface, and delete the existing schedule by clicking the row of the table. Figure 14 shows the screenshot of implemented Feed Machine tab.

% ::Water Supplier tab::
%\subsubsection{Water Supplier tab}
%Similar to the Feed Machine tab, the Water Supplier tab provides two widgets: Device Information widget and Water Supplier Statistics widget. The mechanism for each widget is identical to the Feed Machine tab. Figure 15 shows the screenshot of implemented Water Supplier tab.

%\begin{figure}[htbp]
%\centerline{\includegraphics[width=0.5\textwidth]{./images/water_supplier_ui.png}}
%\caption{Screenshot for Water Supplier tab}
%\label{fig}
%end{figure}

% ::User Settings tab::
%\subsubsection{User Settings tab}
%The User Settings tab provides five widgets: User Information widget, Timezone settings widget, Notification settings widget, E-Mail address settings widget, and WhatsApp settings widget. 
% ::Sign in by token::
%In the User Information widget, users can sign in to the system by entering the token, which is an identifier for the user. After signing in, the user can see the token and username of the current user.
% ::Set timezone::
%To support global users, the Timezone settings widget is included in the tab. As all the times that platform uses and database stores use Coordinated Universal Time (UTC) timezone, converting the time of the user to the UTC is necessary.
% ::Notification option::
%Notification settings widget, E-Mail address settings widget, and WhatsApp settings widget are to control error notification. Users can turn notifications on and off with the Notification settings widget and decide which accounts receive email and WhatsApp messages. Figure 16 shows the screenshot of the implemented User Settings tab.

%\begin{figure}[htbp]
%\centerline{\includegraphics[width=0.5\textwidth]{./images/user_settings_ui.png}}
%\caption{Screenshot for User Settings tab}
%\label{fig}
%\end{figure}

% ::—— EXPERIMENT & TESTING ——::
\section{Experiment and Testing}
Testing is conducted based on the results of the implementation of the Petification IoT Platform. Testing is conducted only on quantitatively verifiable results, and a total of two cases are covered: accuracy testing of the weight measurement and that of the automatic feeding.

% ::Weight measurement testing::
\subsection{Weight measurement testing}
In this test, a total of 8 tests were performed using a load cell sensor in which calibration was completed, and actual weight is compared with the measured weight in each test.

\begin{figure}[htbp]
\centerline{\includegraphics[width=0.5\textwidth]{./images/Calibration_sheet.jpg}}
\caption{Testing result of weight measurement}
\label{fig}
\end{figure}

It was confirmed that the average accuracy was 99.8\%, which resulted in high accuracy. Through this, it can be confirmed that the weight measurement error by the sensor hardly occurs as the result of subsequent testing.

% ::Automatic feeding testing::
\subsection{Automatic feeding testing}
In this test, a total of 10 tests were performed to figure out the automatic feeding feature works as expected. The desired serving amount is compared with the actual serving amount in each test.

\begin{figure}[htbp]
\centerline{\includegraphics[width=0.5\textwidth]{./images/Feeding_sheet.jpg}}
\caption{Testing result of automatic feeding}
\label{fig}
\end{figure}

As a result of the measurement, it was confirmed that 40\% of the highest error was found, and 1.5\% of the lowest error was found. The reason for the error is that the food is often stuck in front of the food gate and the food serving speed is too fast for the load cell to detect. It is a limitation for the feed machine and can be solved if the design is modified correctly.

% ::—— CONCLUSION ——::
\section{Conclusion}
In the present research, a pet-care IoT solution named ‘Petification’ is proposed to take care of users’ pets when they are not at home. The water supplier is supported in the petification to supply water to the pet and report the current amount of the remaining water. The feed machine is also supported to serve a certain amount of food to the pet and report the current amount of the remaining food. The IoT platform for Petification is implemented using Node-RED to take advantage of open-source and flow-based visual programming. Also, the message flow for the Petification is controlled with Eclipse Mosquitto, which is an open-source implementation of the MQTT Protocol. Users for this solution can check the water and food consumption of the pet and serve the food to the pet on the web-based dashboard. The dashboard of the petification also provides device connectivity and the amount of remaining food and water to the user. When the water or food for each device is empty, the user can get notifications for it.

% ::Limitations and future plans::
While all other functionalities are working correctly, the accuracy of serving the exact amount of food to the pet is relatively low. The low accuracy is caused by the design of the food gate and the sensitivity of the load cell. Thus, the future plan can be enhancing the design of the food gate and the sensitivity and stability of the load cell to serve the exact amount of food. Also, attaching another type of device such as a device for taking pictures of the pet can be a possible improvement for Petification.

% ::—— ACKNOWLEDGEMENT ——::
\section{Acknowledgement}
“This research was supported by the MSIT(Ministry of Science and ICT), Korea, under the National Program for Excellence in SW) supervised by the IITP(Institute of Information \& communications Technology Planing \& Evaluation) in 2021”(2021-0-01435)
The authors of this study are grateful to Professor Minsun Lee of Chungnam National University, Professor Eric T. Matson, Professor Anthony H. Smith, and Teaching Assistant Minji Lee of Purdue University for helping us participate in the project.

\begin{thebibliography}{00}
\bibitem{b1} 
M.  Hanson.  “Pet  Industry  Statistics”  spots.com.  https://spots.com/pet-industry-statistics/ (accessed Jan. 25, 2022). 
\bibitem{b2}
Accessed: Feb. 1, 2022. [Online]. Available: https://www.instructables.com/IOT-Pet-Feeder-Using-the-Blynk-Mobile-App-an-ESP82/
\bibitem{b3}
Accessed: Feb. 1, 2022. [Online]. Available: https://www.instructables.com/IoT-Pet-Feeder/
\bibitem{b4}
T. Sangvanloy and K. Sookhanaphibarn, "Automatic Pet Food Dispenser by using Internet of Things (IoT)," 2020 IEEE 2nd Global Conference on Life Sciences and Technologies (LifeTech), Kyoto, Japan, Mar. 10-12, 2020.
\bibitem{b5}
Y. Chen and M. Elshakankiri, "Implementation of an IoT based Pet Care System," 2020 Fifth International Conference on Fog and Mobile Edge Computing (FMEC), Paris, France, Apr. 20-23, 2020.
\bibitem{b6}
Accessed: Feb. 4, 2022. [Online]. Available: https://iotdesignpro.com/projects/google-assistant-controlled-iot-pet-feeder-using-esp8266
\bibitem{b7}
Accessed: Feb. 5, 2022. [Online]. Available: https://create.arduino.cc/projecthub/circuito-io-team/iot-pet-feeder-10a4f3
\bibitem{b8}
Node-RED [Online]. Available: https://nodered.org/about/
\bibitem{b9}
MQTT [Online]. Available: https://mqtt.org
\bibitem{b10} %10
P. N. Vrishanka, P. Prabhakar, D. Shet and K. Rupali, "Automated Pet Feeder using IoT," 2021 IEEE International Conference on Mobile Networks and Wireless Communications (ICMNWC), Tumkur, Karnataka, India, Dec. 3-4, 2021.
\bibitem{b11} %11
R. Nogueira, H. Araújo and D. Prata. (Apr. 2019). Robot Chow: Automatic Animal Feeding with Intelligent Interface to Monitor Pets. International Journal of Advanced Engineering Research and Science. [Online]. Available: https://ijaers.com/detail/robot-chow-automatic-animal-feeding-with-intelligent-interface-to-monitor-pets/
\bibitem{b12} %12
Vania, K. Karyono and I. H. T. Nugroho, "Smart dog feeder design using wireless communication, MQTT and Android client," 2016 International Conference on Computer, Control, Informatics and its Applications (IC3INA), Tangerang, Indonesia, Oct. 3-5, 2016.
\bibitem{b13} %13
Thepihut. https://thepihut.com/blogs/raspberry-pi-roundup/whats-the-difference-between-dc-servo-amp-stepper-motors (accessed Feb. 04, 2022).
\bibitem{b14} %14
Accessed: Feb. 15, 2022. [Online]. Available: https://instrumentationtools.com/load-cell-working-principle/
\bibitem{b15} %15
Accessed: Feb. 19, 2022. [Online]. Available: https://www.seeedstudio.com/blog/2019/11/26/10-things-you-can-do-with-your-hx711-and-load-cell/
\bibitem{b16} %16
Accessed: Feb. 6, 2022. [Online]. Available: https://www.fujielectric.com/products/column/servo/servo\_01.html
\bibitem{b17} %17
N. B. Kamarozaman and A. H. Awang, "IOT COVID-19 Portable Health Monitoring System using Raspberry Pi, Node-Red and ThingSpeak," 2021 IEEE Symposium on Wireless Technology \& Applications (ISWTA), Shah Alam, Malaysia, Aug. 17-17, 2021.
\bibitem{b18} %18
N. Naik, "Choice of effective messaging protocols for IoT systems: MQTT, CoAP, AMQP and HTTP," 2017 IEEE International Systems Engineering Symposium (ISSE), Vienna, Austria, Oct. 11-13, 2017.
\bibitem{b19} %19
OASIS Open [Online]. Available: https://www.oasis-open.org/committees/tc\_home.php?wg\_abbrev=mqtt
\bibitem{b20} %20
Eclipse Mosquitto [Online]. Available: https://mosquitto.org/
\bibitem{b21} %21
Datamation. [Online]. Available: https://www.datamation.com/storage/8-major-advantages-of-using-mysql/
\bibitem{b22} %22
A. Tamboli, “Build Your Own IoT Platform,” in \textit{Apress}, 1st ed, 2019
\bibitem{b23} %23
Node-RED [Online]. Available https://flows.nodered.org/node/node-red-dashboard

\end{thebibliography}
\vspace{12pt}
\end{document}
